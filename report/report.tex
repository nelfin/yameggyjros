\documentclass[a4paper,10pt]{article}
\usepackage{graphicx}
\usepackage[margin=3.0cm]{geometry}
\usepackage[english]{babel}

\usepackage[sc]{mathpazo}
\linespread{1.02} % Increase leading ever so slightly for Palatino
\renewcommand{\sfdefault}{cmbr}
\usepackage[scaled=0.86]{beramono} % Match x-height
\usepackage[T1]{fontenc}

\usepackage{multirow}
\usepackage{booktabs}
\usepackage{listings}
\lstset{language=C, basicstyle={\ttfamily}}

%\usepackage[normalem]{ulem}
%\usepackage{algorithm2e}
%\usepackage{amsfonts}
%\usepackage{amsmath}
%\usepackage{amssymb}
%\usepackage{tabularx}
%\usepackage{color}
%\usepackage{tabto}
%\usepackage{graphicx}
%\usepackage{hyperref}
%\usepackage{float}
%\usepackage{epstopdf}
%\usepackage{caption}
%\usepackage{subfigure}

\newcommand\meggyjr{Meggy Jr.\ RGB}

\begin{document}

\begin{titlepage}
 
\begin{center}

\textsc{\Large Research School of Computer Science}\\[0.5cm]
\textsc{\Large College of Engineering and}\\[0.2cm]
\textsc{\Large Computer Science}\\[0.5cm]


 
\vspace{1.4cm}

\hrule

\vspace{1.4cm}

{ \huge \bfseries An Operating System for the \meggyjr}
\vspace{1.4cm}
\hrule
\vspace{1.0cm}

\textsc{\large COMP3300 --- Operating Systems Implementation}

\vspace{1.0cm}

\vspace{1.0cm}
{ \Large \bfseries Andrew Haigh --- u4667010}
\begin{itemize}
\item Screen Library
\item Sound Library
\item Game implementation
\item Multithread Library
\end{itemize}
\vspace{1.0cm}
{ \Large \bfseries Joshua Nelson --- u4850020}
\begin{itemize}
 \item EEPROM Library
 \item Keys Library
 \item Serial Library (Using Eric McCreath's example code)
 \item Multithread Library
\end{itemize}
\vfill
\begin{abstract}
A rudimentary operating system for the \meggyjr, a handheld device based on
the Atmel AVR architecture. A library useful to game developers was created,
including functions for interacting with the screen, sound, keys, memory, I/O,
and multithreading.
\end{abstract}
 
\end{center}

\end{titlepage}

\section*{Application Programmer's Library}

In general, the APL is divided into two parts: the \texttt{dev/} modules which
expose the low level interface for using the \meggyjr, and the \texttt{lib/}
modules which provide convenience functions and higher level abstractions of
the pieces of hardware.

\subsection*{Screen Library}
\label{sub:Screen Library}

The screen library consists of the \texttt{dev/leds} and \texttt{lib/screen}
modules. The lower level interface exports 1 byte for the auxiliary LEDs and a
shared memory region of 64 pixels that can be manipulated directly by the user
if they know what they are doing. \texttt{lib/screen}, in concert with
\texttt{lib/colours}, provides methods for filling the screen or a single
column with a specific colour. The \texttt{dev/leds} module additionally sets
up the screen refresh interrupt vector and defines the internal pixel
representation.

The screen is completely refreshed at a rate of approximately 124\thinspace
Hz. Precisely, the screen is redrawn by a clear on compare interrupt utilising
\texttt{Timer2} that fires every 2,688 cycles (every 84 cycles with a
prescaling of $\frac{1}{32}$). A total of 48 firings are needed to completely
refresh the screen, given 8 columns with 3 colour luminosity times 2
brightness passes.  This gives a precise refresh rate of:

\[
  \frac{16,000,000}{32} \div 84 \div (8 \times 3 \times 2) \approx
  124.0079\,\mathrm{Hz}.
\]

Internally, a pixel is represented as a packed 7 bit structure. This is
specified as a C \lstinline!union! to allow aliasing between the bit
representation and a for convenience. The bit
representation of a pixel is as follows:

\begin{center}
\begin{tabular}{ccccccccc}
\toprule
Colour Code & \multicolumn{8}{c}{Binary Representation}\tabularnewline
\midrule
 \multirow{2}{*}{\#0055ff} & 0 & 1 & 1 & 1 & 0 & 1 & 0 & 0\tabularnewline
 &  & V & B & B & G & G & R & R\tabularnewline
\bottomrule
\end{tabular}
\end{center}
where V is brightness, and R, G, B are red, green, and blue respectively.

\subsection*{Sound Library}
\label{sub:Sound Library}

The sound library consists of the two modules \texttt{dev/sound} and
\texttt{lib/music}. The sound library is fairly simple and requires minimal
CPU time as it is not interrupt driven. Instead, \texttt{Timer1} (which
features 16-bit counters) is used to drive a PWM signal that is fed to the
piezo buzzer, producing accurate tones. The \texttt{lib/music} module provides
a more convenient interface allowing the user to specify a note name and
duration in milliseconds. Tones stop playing automatically via a duration
counter in the screen refresh interrupt.

\subsection*{Keys Library}
The keys library defines names for each of the buttons, and allows a user to
check the specified key is down, wait for a specific key to be pressed, or to
return the next key that is pressed. These functions were the ones deemed
useful for an application developer.  The functions were implemented with busy
loops, which work well with the pre-emptive multithreading library. An
alternative would be to create a method that used interrupts for key presses,
which has the advantage of avoiding the busy loops required for this
implementation.

\subsection*{Serial Library}
The serial library allows data to be sent over the Meggy Jr.'s serial cable,
which is useful for debugging. Functions were created for sending strings and
variables via the serial cable to a connected computer's terminal. This was
used during the creation and debugging of the other code.

\section*{Multithreading}
%%% Pretty sure this is more than half a page, feel free to delete rubbish
%%% lines. It's 2:23AM right now and I could easily have written junk.

Our multithreading implementation is a pre-emptive multithreading library,
which allows parallel execution of two `threads', implemented as functions.
The maximum number of simultaneous threads was specified to be two for
simplicity, but it the implementation for $n$ threads would be similar.

The function \texttt{execute\_parallel} is the user's main interface with the
library. It takes two functions as it's arguments - the functions that we are
to run in parallel. It saves the addresses of these functions, and begins to
execute one of them, while setting a timer (\texttt{Timer0}), and enabling an
interrupt when the timer fires. When this occurs, all of the registers are
pushed onto the stack. The library then saves the stack pointer in kernel
space, and moves the stack pointer to be half way down the stack, which is the
starting point for our second thread.  The interrupt then returns control to
the second thread by modifying the return address on the stack.

At this point, we continue until we reach another interrupt, then push the
registers back onto the stack, save the stack pointer, and return to the first
thread's stack pointer. We pop the registers back to their appropriate places,
and continue execution of this thread's code from where we left off.  This
loop continues indefinitely. Since the context switches are triggered
regularly, and we alternate between the two processes fairly, our scheduler
implemented was the round-robin scheduler.

Several simplifications were made in this implementation. It was assumed that
threads do not terminate, and so parallel should contain some infinite loop to
prevent them from completing execution. An extension could be to add a method
of terminating a thread, which notified the context switching library of this
change.  The library is also not capable of running more than two threads
simultaneously. This modification simplifies the implementation of the library
substantially.
It should also be noted that the second thread's stack starts at half way down
the actual stack. If the first thread used up too much stack space (more than
half of the total stack), it would begin to overwrite the second thread's
stack space. An extension to the library could be to add a protection system
to the memory regions.

\section*{Simple Game}

As a demonstration, a simplified clone of the classic block smashing game
Breakout was developed utilising the full features of the APL. The game
features simple animations and sound effects, as well as background music.
Sound effects can be played over the background music as the music is
scheduled by a separate thread to the main game loop.

\section*{Extension: EEPROM Library}
A simple library that interfaced with the \texttt{avr/eeprom.h} header file
was created. The way the user would interact with this library are via the
\texttt{persistent\_memory} array, a 1KB array of data, and through the saving
and loading functions. However, this array is not always consistent with the
actual EEPROM memory: save and load functions are provided for synchronising
this array with the EEPROM memory. An alternative approach would be
synchronisation when modifying this array, or synchronisation at set time
intervals. These approaches have the disadvantage of not letting the user
control when accesses to EEPROM memory occurs, which is important, as the
EEPROM memory is only specified to be error free for up to 100,000 reads and
writes, which is a low amount.

Another feature of the library is the ability to take and load screenshots.
This interacts with the Screen Library well, as it only needs to copy screen
data from the \texttt{rgb\_screen} array to the \texttt{persistent\_memory}
array, and then save the array to EEPROM memory (load and transfer in the
opposite direction for loading screenshots).
One simplification that was made in the EEPROM memory implementation was the
addressing: integer addresses are used for `file' locations, and users must
be careful to avoid overwriting their own data. An improvement that could be
made would be to implement a file table that allowed file names for accessing
and storing data.

\appendix

\section{\texttt{dev/}}

\subsection{\texttt{dev/keys.h}}
\lstinputlisting{../dev/keys.h}

\subsection{\texttt{dev/keys.c}}
\lstinputlisting{../dev/keys.c}

\subsection{\texttt{dev/leds.h}}
\lstinputlisting{../dev/leds.h}

\subsection{\texttt{dev/leds.c}}
\lstinputlisting{../dev/leds.c}

\subsection{\texttt{dev/multithread.h}}
\lstinputlisting{../dev/multithread.h}

\subsection{\texttt{dev/multithread.c}}
\lstinputlisting{../dev/multithread.c}

\subsection{\texttt{dev/persistent\_memory.c}}
\lstinputlisting{../dev/persistent_memory.c}

\subsection{\texttt{dev/serial.h}}
\lstinputlisting{../dev/serial.h}

\subsection{\texttt{dev/serial.c}}
\lstinputlisting{../dev/serial.c}

\subsection{\texttt{dev/sound.h}}
\lstinputlisting{../dev/sound.h}

\subsection{\texttt{dev/sound.c}}
\lstinputlisting{../dev/sound.c}

\section{\texttt{lib/}}

\subsection{\texttt{lib/delay.c}}
\lstinputlisting{../lib/delay.c}

\subsection{\texttt{lib/random.c}}
\lstinputlisting{../lib/random.c}

\subsection{\texttt{lib/music.h}}
\lstinputlisting{../lib/music.h}

\subsection{\texttt{lib/music.c}}
\lstinputlisting{../lib/music.c}

\subsection{\texttt{lib/colours.h}}
\lstinputlisting{../lib/colours.h}

\subsection{\texttt{lib/screen.c}}
\lstinputlisting{../lib/screen.c}

\section{\texttt{breakmeggy.c}}

\lstinputlisting{../breakmeggy.c}

\section{\texttt{makeflash}}

\lstinputlisting[language=bash]{../makeflash}

\end{document}
